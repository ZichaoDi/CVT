% LaTeX Curriculum Vitae Template
%
% Copyright (C) 2004-2009 Jason Blevins <jrblevin@sdf.lonestar.org>
% http://jblevins.org/projects/cv-template/
%
% You may use use this document as a template to create your own CV
% and you may redistribute the source code freely. No attribution is
% required in any resulting documents. I do ask that you please leave
% this notice and the above URL in the source code if you choose to
% redistribute this file.

\documentclass[letterpaper]{article}

\usepackage{hyperref}
\usepackage{geometry}

% Comment the following lines to use the default Computer Modern font
% instead of the Palatino font provided by the mathpazo package.
% Remove the 'osf' bit if you don't like the old style figures.
\usepackage[T1]{fontenc}
\usepackage[sc,osf]{mathpazo}

\title{\bf  Personal Statement}

\maketitle
\centerline{\bf  Zichao Di}


\begin{document}


I started my PhD study in  2010 and meanwhile, work as graduate research assistant in Operation Research. 
My research is in the area of Applied Mathematics with main focus on the numerical analysis, especially in the field of  'Centroid Voronoi Tessellation' known as 'CVT'. Before being a GRA in OR department, my research was mainly about implementing the algorithm involved with 'CVT' such as Lloyd method and multi-level Lloyd. My current research is focus on Multigrid Optimization. 


During the passed one year,  I studied the basic and advanced knowledge of Optimization. Combine these background and the instruction from my advisor, I have a very deep understanding of Multigrid Optimization. Finally, I successfully applied this brand new method 'MG/OPT'  to  'CVT' resulting in a dramatically improved convergence factor compared to the other methods together with my advisors: Stephen Nash & Maria Emelinaenko.


As a big honor, I am invited to give a talk in the minigymposium of ICIAM 2011 -- 7th International Congress on Industrial and Applied Mathematics: 'Recent Advances in Studies and Applications of Centroidal Voronoi Tessellations - Part II of II'. My talk title is 'Novel Optimization - based Multilevel CVT Construction Algorighm'. This minigymposium is organized by Lili Ju and  Maria Emelianenko. 

Therefore, I plan to apply for the travel support funds provided by a grant from the U.S.National Science Foundation.


















\end{document}